
\documentclass[a4paper,10pt,twocolumn]{article}
\usepackage{luatexja}
\usepackage[top=10mm,bottom=10mm,left=10mm,right=10mm]{geometry}
\usepackage{amsmath,amssymb}
\usepackage{multicol}
\usepackage{needspace}

% Clean box layout matching the reference image style
% Single frame, question number and text inside, empty space below.
\newsavebox{\myqbox}
\newenvironment{qbox}{%
  \begin{lrbox}{\myqbox}%
  % Subtract framesep and rule to fit exactly in column
  \begin{minipage}{\dimexpr\linewidth-2\fboxsep-2\fboxrule\relax}
  \setlength{\parskip}{5pt}
}{%
  \end{minipage}%
  \end{lrbox}%
  \par\noindent
  \fbox{\usebox{\myqbox}}%
  \par\vspace{1em} % Space between questions
}

% Answer Box is just vertical space now, no border
\newcommand{\answerbox}[1]{
  \vspace{#1}
}

\begin{document}



\begin{needspace}{4cm}
\begin{qbox}
\textbf{(1)} 次の式を展開してください: $(x + 3)(x + 2)$
\answerbox{3cm}
\end{qbox}
\end{needspace}
        
\begin{needspace}{4cm}
\begin{qbox}
\textbf{(2)} 次の式を展開してください: $(2x - 1)(x + 4)$
\answerbox{3cm}
\end{qbox}
\end{needspace}
        
\begin{needspace}{4cm}
\begin{qbox}
\textbf{(3)} 次の式を展開してください: $(x - 5)^2$
\answerbox{3cm}
\end{qbox}
\end{needspace}
        
\begin{needspace}{4cm}
\begin{qbox}
\textbf{(4)} 次の式を展開してください: $(3x + 2)(x - 3)$
\answerbox{3cm}
\end{qbox}
\end{needspace}
        
\begin{needspace}{4cm}
\begin{qbox}
\textbf{(5)} 次の式を展開してください: $(2x + 3)^2$
\answerbox{3cm}
\end{qbox}
\end{needspace}
        
\newpage
\section*{解答}
\begin{itemize}
\item[(1)] \textbf{[expansion_polynomials]} \quad $$x^2 + 5x + 6$$
        \begin{itemize}\item[\textbf{解説}] まず、分配法則を用いて展開します。$(x + 3)(x + 2) = x(x + 2) + 3(x + 2) = x^2 + 2x + 3x + 6 = x^2 + 5x + 6$\item[\textbf{ヒント}] 分配法則を使って各項を展開しましょう。\item[\textbf{注意}] 間違えて、$x^2 + 6x + 6$と計算してしまうことがあります。\end{itemize}\vspace{0.5em}
\item[(2)] \textbf{[expansion_polynomials]} \quad $$2x^2 + 7x - 4$$
        \begin{itemize}\item[\textbf{解説}] 分配法則を使います。$(2x - 1)(x + 4) = 2x(x + 4) - 1(x + 4) = 2x^2 + 8x - x - 4 = 2x^2 + 7x - 4$\item[\textbf{ヒント}] 各項に分けて分配法則を適用しましょう。\item[\textbf{注意}] 負の符号を忘れてしまうことがあります。\end{itemize}\vspace{0.5em}
\item[(3)] \textbf{[expansion_polynomials]} \quad $$x^2 - 10x + 25$$
        \begin{itemize}\item[\textbf{解説}] 二乗の公式を使います。$(x - 5)^2 = (x - 5)(x - 5) = x^2 - 5x - 5x + 25 = x^2 - 10x + 25$\item[\textbf{ヒント}] 二乗公式 $(a - b)^2 = a^2 - 2ab + b^2$ を使いましょう。\item[\textbf{注意}] $x^2 - 25$と計算してしまうことがあります。\end{itemize}\vspace{0.5em}
\item[(4)] \textbf{[expansion_polynomials]} \quad $$3x^2 - 7x - 6$$
        \begin{itemize}\item[\textbf{解説}] 分配法則を使います。$(3x + 2)(x - 3) = 3x(x - 3) + 2(x - 3) = 3x^2 - 9x + 2x - 6 = 3x^2 - 7x - 6$\item[\textbf{ヒント}] 各項を順番に展開しましょう。\item[\textbf{注意}] 符号に注意してください。間違えて符号を変えることがあります。\end{itemize}\vspace{0.5em}
\item[(5)] \textbf{[expansion_polynomials]} \quad $$4x^2 + 12x + 9$$
        \begin{itemize}\item[\textbf{解説}] 二乗の公式を使います。$(2x + 3)^2 = (2x + 3)(2x + 3) = 4x^2 + 6x + 6x + 9 = 4x^2 + 12x + 9$\item[\textbf{ヒント}] 二乗公式 $(a + b)^2 = a^2 + 2ab + b^2$ を使いましょう。\item[\textbf{注意}] 中間項を$6x$と計算してしまうことがあります。\end{itemize}\vspace{0.5em}\end{itemize}
    

\end{document}
     